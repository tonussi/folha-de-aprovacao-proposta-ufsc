\documentclass{ufsc-thesis}

\usepackage{pgfplots, pgfplotstable}
\usepackage{xcolor}

\usepackage{calc, ifthen, tikz}
\usetikzlibrary{fadings}

\usepackage{cmap}
\usepackage[utf8]{inputenc}
\usepackage[T1]{fontenc}

\usepackage{amsmath}
\usepackage{amssymb}
\usepackage{graphicx}
\usepackage{color}
\usepackage{listings}
\usepackage{hyperref}
\usepackage{wrapfig}

\begin{document}

\begin{center}
  \Large \textbf{FOLHA DE APROVAÇÃO DE PROPOSTA DE TCC}\\[5px]

\begin{flushleft}
    \resizebox{0.995\textwidth}{!}{
    \begin{tabular}{|l|l|}
      \hline \textbf{Acadêmico} & Lucas Pagotto Tonussi \\
      \hline \textbf{Título do trabalho (subtítulo)} & Proposta de Implementação de Replicação por Máquina de Estados em Arquiteturas de Microsserviços \\
      \hline \textbf{Curso} & Ciência da Computação ou Sistemas de Informação /INE/UFSC \\ 
      \hline \textbf{Área de Concentração} & Arquitetura de Sistemas de Computação \\
      \hline 
    \end{tabular}
  }
\end{flushleft}
{
\small
\begin{flushleft}
\quad \textbf{Instruções para preenchimento pelo \underline{ORIENTADOR DO TRABALHO}:}

\qquad - Para cada critério avaliado, assinale um X na coluna SIM apenas se considerado aprovado.

\qquad - Caso contrário, indique as alterações necessárias na coluna Observação.
\end{flushleft}
}

  \begin{flushleft}
  \resizebox{0.991\textwidth}{!}{
  \begin{tabular}{|l|c|c|c|c|c|}
  \hline  & \multicolumn{3}{c}{Aprovado}  &  & Observação \\ 
  \hline \textbf{Critérios} & Sim & Parcial & Não & Não se aplica & \\
  \hline \parbox[c][50px]{450px}{1. O trabalho é adequado para um TCC no
  CCO/SIN (relevância / abrangência)?}  & & & & & \\
  \hline \parbox[c][50px]{450px}{2. O titulo do trabalho é adequado?}  & & & & 
  & \\
  \hline \parbox[c][50px]{450px}{3. O tema de pesquisa está claramente
  descrito?} & & & & & \\
  \hline \parbox[c][50px]{450px}{4. O problema/hipóteses de pesquisa do
  trabalho está claramente identificado?} & & & & & \\
  \hline \parbox[c][50px]{450px}{5. A relevância da pesquisa é justificada?}  &
  & & & & \\
  \hline \parbox[c][50px]{450px}{6. Os objetivos descrevem completa e
  claramente o que se pretende alcançar neste
  trabalho?}  & & & & & \\
  \hline \parbox[c][50px]{450px}{7. É definido o método a ser adotado no
  trabalho? O método condiz com os objetivos e
  é adequado para um TCC?}  & & & & & \\
  \hline \parbox[c][50px]{450px}{8. Foi definido um cronograma coerente com
  o método definido (indicando todas as
  atividades) e com as datas das entregas
  (p.ex. Projeto I, II, Defesa)?}  & & & & & \\
  \hline \parbox[c][50px]{450px}{9. Foram identificados custos relativos à
  execução deste trabalho (se houver)? Haverá
  financiamento para estes custos?}  & & & & & \\
  \hline \parbox[c][50px]{450px}{10. Foram identificados todos os envolvidos
  neste trabalho?}  & & & & & \\
  \hline \parbox[c][50px]{450px}{11. As formas de comunicação foram
  definidas (ex: horários para orientação)?}  & & & & & \\
  \hline \parbox[c][50px]{450px}{12. Riscos potenciais que podem causar
  desvios do plano foram identificados?}  & & & & & \\
  \hline \parbox[c][80px]{450px}{13. Caso o TCC envolva a produção de um
  software ou outro tipo de produto e seja desenvolvido também como uma
  atividade realizada numa empresa ou laboratório, consta da proposta uma
  declaração (Anexo 3) de ciência e concordância com a entrega do código fonte
  e/ou documentação produzidos?} & & & & & \\
  \hline
  \end{tabular}}
  \end{flushleft}

  \begin{flushleft}
\resizebox{0.996\textwidth}{!}{
  \begin{tabular}{|c|c|c|c|}
  \hline \textbf{Avaliação} & \multicolumn{2}{c}{ $\Box$ Aprovado \makebox[3in]{ } $\Box$ Não Aprovado } & \\
  \hline
  \hline
  \textbf{Professor Responsável}: & Odorico Machado Mendizabal & \makebox[3in]{ } &  \makebox[3in]{ } \\
  \hline
  \end{tabular}
}
  \end{flushleft}

\end{center}

\end{document}
